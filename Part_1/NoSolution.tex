\chapter{When Numerical Problems Have No Solution} \label{chap:NoSolution}

There are a variety of problems involving mathematics.  A \emph{mathematics} problem is defined by in terms of mathematical notation with a specified outcome required, e.g., 
\begin{gather}\label{eq:mathzero}
    \text{given } f \in C[a,b],  \quad \text{find } x \text{ satisfying } f(x) = 0, \\
    \label{eq:uniIntegral}
    \text{given } f \in C[a,b],  \quad \text{find } \int_a^b f(x) \dif x.
\end{gather} 
One may specify necessary and sufficient conditions under which solutions exist and/or existence.  There exists a solution to \eqref{eq:mathzero} if $f(a) \le 0 \le f(b)$.  This solution is unique if $f$ is strictly increasing.  There exists a unique solution to \eqref{eq:uniIntegral}.

If \eqref{eq:mathzero} appeared on a secondary school test, $f$ would likely be expressed in terms of elementary functions and the expected answer would be an analytic one.  The same might be the case for \eqref{eq:uniIntegral} on a calculus test.  The conditions under which this is possible are studied by those developing symbolic computation software.  

Before arriving at a mathematical statement of a problem, one might be faced with the problem posed in words.  Interpretation of the words and a certain amount of mathematical modeling might be required to reduce the problem to a mathematical one.  The determination of whether or not a solution is found might be judged on practical or subjective grounds.

The numerical problems introduced in \cref{chap:ProbAlgo} differ from mathematical problems in that they require an approximate solution satisfying an error criterion.  They take the following form:
\iffalse
\begin{NumProblem}[Generic Problem]
\label{prob:generalProblem}
\problemspecs{set of functions $\cf$ \\ 
solution operator $\sol: \cf \to \cg$}
{black-box function $f \in \cf$ \\ error tolerance (vector) $\veps$}
{$\out \in \cg$ such that \\ \qquad $\crit(\out,\sol(f),\veps) = \true$}
\end{NumProblem}
\fi
Here $\cg$ is the set of possible solutions.  It corresponds to $[a,b]$ in mathematical problem \eqref{eq:mathzero} and $\reals$ in mathematical problem \eqref{eq:uniIntegral}.  The wording ``within $\varepsilon$'' is interpreted in light of the specific error criterion.
In defining the problem, it is important that $\cf$ be defined large enough so that the successful algorithm is widely applicable.  However, it is also important that $\cf$ be constrained so that a successful algorithm exists.  

The following meta-theorem provides condition under which \cref{prob:generalProblem} is unsolvable.

\begin{theorem}[When No Solution Exists] \label{thm:impossible}
Suppose that a numerical problem takes the form of \cref{prob:generalProblem}.  Fix some acceptable tolerance, $\veps$.  If for every $n \in \naturals$ and every possible $\vx_1, \ldots, \vx_n$ in the union of the domains of the functions in $\cf$, one can identify two distinct fooling functions $f_1, f_2 \in \cf$ with the same function data but quite different solutions, i.e.,
\begin{gather} \label{eq:bumpfunctionsmatch}
    f_1(\vx_1) = f_2(\vx_1), \ldots,  f_1(\vx_n) = f_2(\vx_n), \text{ and} \\
    \label{eq:solutionsfar}
    \{\out \in \cg : \crit(\out,\sol(f_1),\veps) \land \crit(\out,\sol(f_2),\veps) = \true \} =\emptyset,
\end{gather}
then \cref{prob:generalProblem} has no solution.  That is, no algorithm can solve this problem.
\end{theorem}